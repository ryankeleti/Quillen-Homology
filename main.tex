\documentclass[10pt,reqno]{amsart}

\usepackage[T1]{fontenc}
\usepackage[left=0.5in,right=0.5in,top=0.5in,bottom=0.5in]{geometry}
\usepackage{mathtools}
\usepackage{stmaryrd}
\usepackage{hyperref}
\usepackage{tikz}
\usepackage{tikz-cd}
\usepackage{cite}
\usepackage{url}
\usepackage{sseq}
%\renewcommand{\rmdefault}{pplx}
%\usepackage{eulervm}
\usepackage[charter]{mathdesign}
\hypersetup{
  colorlinks,
  allcolors=[rgb]{0,0.45,0}
}
\binoppenalty=\maxdimen
\relpenalty=\maxdimen
\allowdisplaybreaks[1]
%\renewcommand\refname{References.}
\title{Homology of commutative rings}
\author{D.~Quillen}

\DeclareMathOperator{\Hom}{Hom}
\DeclareMathOperator{\Der}{Der}
\DeclareMathOperator{\Tor}{Tor}
\DeclareMathOperator{\Ext}{Ext}
\DeclareMathOperator{\Ann}{Ann}
\DeclareMathOperator{\Spec}{Spec}
\DeclareMathOperator{\Exalcomm}{Exalcomm}
\DeclareMathOperator{\Ho}{Ho}
\DeclareMathOperator{\gr}{gr\,}
\DeclareMathOperator{\id}{id}
\DeclareMathOperator{\im}{im}
\DeclareMathOperator{\coker}{coker}

\makeatletter
\newenvironment{nenv}[1]{
\par\medskip\noindent\textbf{#1}.\,\itshape
}
\makeatother

\makeatletter
\newenvironment{env}[2]{
\par\medskip\noindent\textbf{#1}~\textbf{#2}.\,\itshape
}
\makeatother

\makeatletter
\newenvironment{envr}[2]{
\par\medskip\noindent\textbf{#1}~\textbf{#2}.\,\rmfamily
}
\makeatother

\makeatletter
\newenvironment{prop}[1]{
\par\medskip\noindent\textbf{Proposition}~\textbf{#1}.\,\itshape
}
\makeatother

\newcommand{\lra}{\longrightarrow}
\newcommand{\Lra}{\Longrightarrow}
\newcommand{\cat}{\mathcal}
\newcommand{\A}{\cat{A}}
\newcommand{\sA}{\mathrm{s}\A}
\newcommand{\C}{\cat{C}}
\newcommand{\M}{\cat{M}}
\newcommand{\R}{\cat{R}}
\newcommand{\sR}{\mathrm{s}\R}
\newcommand{\T}{\cat{T}}
\newcommand{\Set}{\mathrm{Set}}
\renewcommand{\L}{\mathbf{L}}
\renewcommand{\H}{\mathrm{H}}
\newcommand{\D}{\mathrm{D}}
\newcommand{\Mod}[1]{\mathrm{Mod}_{#1}}
\newcommand{\Alg}[1]{\mathrm{Alg}_{#1}}
% change this to D (original) or \mathrm{Diff} or whatever you like
\newcommand{\Diff}{\Omega}
\newcommand{\LD}{\L\Diff}

\begin{document}
\maketitle

\section*{Chapter~I.}
\subsection*{\textsection 1. Differentials and derivations}

All rings are commutative, associative, with identity and all modules are unitary unless
stated otherwise. $\R$ denotes the category of rings. Also all diagrams are understood
to be commutative.

Let $A$ be a ring and let $\C=A\backslash\R$ be the category of $A$-algebras.
An object of $\C$ is a ring $B$ together with a map $i_B:A\to B$ of rings,
and a map $f:B\to B'$ in $\C$ is a map of rings such that $fi_B=i_{B'}$. $\C$
is a category of universal algebras so the general theory of [HA, \textsection 5, p.~5.14]
applies to define cohomology groups for an object $B$
of $\C$ with values in a ``$B$-module,'' where ``$B$-module'' in the general theory
means an abelian group object in $\C/B$. We show in this section that this notion of
$B$-module coincides with the usual one. Also we calculate the abelianization functor
on $\C/B$ in terms of differentials.

Let $B$ denote a fixed $A$-algebra and let $\C/B$ be the category of $A$-algebras
over $B$. Here ``over'' is used as in category theory, so that an object of $\C/B$
is an $A$-algebra together with a map $u_X:X\to B$ of $A$-algebras. If $M$ is a
$B$-module, let $B\oplus M$ be the $A$-algebra over $B$ with
\[
  (b\oplus m)(b'\oplus m')=bb'\oplus(bm'+b'm),
\]
\[
  i_{B\oplus M}(a)=i_B(a),
\]
\[
  u_{B\oplus M}(b\oplus m)=b.
\]
If $\theta:Y\to B\oplus M$ is a map in $\C/B$, then $\theta y=u_Y y\oplus Dy$ where
$D:Y\to M$ satisfies
\[
  Di_Y(a)=0,
\]
\[
  D(yy')=u_Y(y)Dy'+u_Y(y')Dy.
\]
In other words $D$ is an $A$-derivation of $Y$ with values in $M$ considered as a
$B$-module via $u_Y$. Conversely given such a $D$ be obtain a $\theta$ and therefore
\[
  \Hom_{\C/B}(Y,B\oplus M)\simeq\Der(Y/A,M).\tag{1.1}
\]
This is an isomorphism of functors of $Y$. As $\Der(Y/A,M)$ is an abelian group under
addition we see that $B\oplus M$ is an abelian group object of $\C/B$.

Recall that an $H$-object ($H$ for H. Hopf) in a category is an object endowed with
an operation having a two-sided identity. An $H$-object of $\C/B$ is therefore an
object $X$ together with maps
\[
  \varepsilon:B\lra X,
\]
\[
  \mu:X\times_B X\lra X,
\]
such that
\[
  \mu\circ(\varepsilon u_X,\id_X)=\mu\circ(\id_X,\varepsilon u_X)=\id_X.\tag{1.2}
\]
For the abelian group object $B\oplus M$ one calculates that
\[
  \varepsilon(b)=b\oplus 0,
\]
\[
  \mu(b\oplus m,b\oplus m')=b\oplus(m+m').\tag{1.3}
\]

\begin{prop}{1.4}
Any $H$-object of $\C/B$ is isomorphic to $B\oplus M$ with multiplication (1.3) for
some $B$-module $M$. In particular any $H$-object is an abelian group object.
\end{prop}
\begin{proof}
Given an $H$-object $X$, let $M=\ker(u_X:X\to B)$ considered as a $B$-module via
$\varepsilon_X:B\to X$. If $x,y\in M$, then by (1.2)
\[
  \mu(x,0)=x,\quad\mu(0,y)=y
\]
so as $\mu$ is a homomorphism
\[
  xy=\mu(x,0)\mu(0,y)=\mu((x,0)(0,y))=\mu(0,0)=0.
\]
Thus $M$ has the zero multiplication. It follows easily that the map $B\oplus M\to X$
given by $b\oplus m\mapsto\varepsilon_X b+m$ is an isomorphism in $\C/B$. Abbreviating
$\varepsilon_X$ to $\varepsilon$ we have
\begin{align*}
  \mu(\varepsilon b+m,\varepsilon b+m')
  &=\mu(\varepsilon b,\varepsilon b)+\mu(m,0)+\mu(0,m')\\
  &=\varepsilon b+m+m',
\end{align*}
which shows that the multiplication $\mu$ on $X$ is the same as (1.3) on $B\oplus M$.
\end{proof}

Let $(\C/B)_\mathrm{ab}$ be the category of abelian group objects in $\C/B$ and let
$\M_B$ be the abelian category of $B$-modules. From 1.4 we obtain
\begin{prop}{1.5}
There is an equivalence of categories
\[
  B\oplus M\longmapsfrom M
\]
\[
  (\C/B)_\mathrm{ab}\simeq\M_B
\]
\[
  X\longmapsto\ker u_X.
\]
In particular $(\C/B)_\mathrm{ab}$ is an abelian category.
\end{prop}

If $Y$ is an $A$-algebra, let $\Diff_{Y/A}$ be the $Y$-module of $A$-differentials of $Y$.
There is a canonical $A$-derivation $d:Y\to\Diff_{Y/A}$ such that
\[
  \Hom_{\M_Y}(\Diff_{Y/A},N)\simeq\Der(Y/A,N)
\]
\[
  \theta\longmapsto\theta\circ d
\]
for any $Y$-module $N$. From 1.1 we have
\[  
  \Hom_{\C/B}(Y,B\oplus M)\simeq\Der(Y/A,M)
  \simeq\Hom_{\M_Y}(\Diff_{Y/A},M)\simeq\Hom_{\M_B}(\Diff_{Y/A}\otimes_Y B,M).\tag{1.6}
\]
Thus

\begin{prop}{1.7}
With respect to the equivalences of categories of 1.5 we have
\[
  \Diff_{Y/B}\otimes_Y B\longmapsfrom Y
\]
\[
  \begin{tikzcd}
  {\M_B\simeq(\C/B)_\mathrm{ab}}\ar[r,"i"',shift right]
  & {\C/B}\ar[l,"\mathrm{ab}"',shift right]
  \end{tikzcd}
\]
\[
  M\longmapsto B\oplus M,
\]
where $i$ is the natural faithful functor and $\mathrm{ab}$ is the abelianization functor,
the left adjoint of $i$.
\end{prop}

From now on we will identify $(\C/B)_\mathrm{ab}$ and $\M_B$ by the equivalence of
Prop. 1.5. 1.7 shows that $Y\mapsto\Diff_{Y/A}\otimes_Y B$ is identified with the
abelianization functor on $\C/B$.

\subsection*{\textsection 2. Homology and cohomology}

The $q$-th cohomology group of the $A$-algebra $B$ with values in the abelian group
object $B\oplus M$ of $\C/B$ [HA, II, p.~5.14] will be denoted
$\D^q(B/A,M)$ and called simply the $q$-th cohomology group of the $A$-algebra $B$
with values in the $B$-module $M$. According to Theorem 5 (loc. cit.) this may be
defined is two different but equivalent ways---as shead cohomology from a
Grothendieck topology and by (semi-)simplicial resolutions. Both definitions for
$\D^q(B/A,M)$ will be given in this section. The former will be used in globalizing
the definition to preschemes and the latter leads to a notion of homology for the
$A$-algebra $B$.

(2.1). Let $\T$ be the Grothendieck topology [GT] whose underlying
category is $\C/B$ and where a covering of $Y$ is a family consisting of a single map
$Z\to Y$ which is set-theoretically surjective. Representable functors are sheaves for
$\T$ hence by 1.1 $Y\mapsto\Der(Y/A,M)$ is a sheaf of $B$-modules on $\T$.
The first definition for the cohomology of $B$ with values in $M$ is
\[
  \D^q(B/A,M)\simeq\H_\T^q(B,\Der(-/A,M))\tag{2.2}
\]
where the right side denotes sheaf cohomology for the topology $\T$.

We will give other definitions for $\D^q$ in 2.14. It is first necessary to state some
facts about simplicial objects.

(2.3). If $X$ is a simplicial object in an abelian category, its homology in
dimension $q$, denoted $\H_q(X)$, is defined to be the homology in dimension $q$ of
the chain complex contructed from $X$ with differential $d=\sum_i(-1)^i d_i$. By the
normalization theorem this is the same as $\H_q(NX)$ where $NX$ is the normalized
subcomplex of $X$. Hence when $X$ is a simplicial abelian group, $\H_q(X)$ is the
$q$-th homology group of $X$ in the sense of Moore.

(2.4). If $f:X\to Y$ and $i:U\to V$ are maps in a category, we say that $f$ has the
right lifting property (RLP) with respect to $i$ and that $i$ has the left lifting
property (LLP) with respect to $f$ if given any commutative square of solid arrows
\[
  \begin{tikzcd}
    U\ar[r]\ar[d,"i"'] & X\ar[d,"f"]\\
    V\ar[r]\ar[ur,dotted] & Y
  \end{tikzcd}
\]
a dotted arrow exists such that the whole diagram is commutative. Let $\Delta(n)$ be the
``standard $n$-simplex' simplicial set and let $i_n:\Delta(n)\to\Delta(n)$ be the
inclusion of its $(n-1)$-skeleton. The following proposition characterizes those maps
of simplicial abelian groups which as trivial fibrations in the sense of [HA].

\begin{prop}{2.5}
The following assertions are equivalent for a map $f:X\to Y$ of simplicial abelian
groups.
\begin{itemize}
  \item[(i)] $f$ is surjective (in each dimension) and $\H_\ast(f):\H_\ast(X)\simeq\H_\ast(Y)$.
  \item[(ii)] As a map of simplicial sets $f$ has the RLP with respect to
        $i_n:\Delta(n)\to\Delta(n)$ for all $n\geq 0$.
  \item[(iii)] As a map of simplicial sets $f$ has the RLP with respect to any injective
        (in each dimension) map of simplicial sets.
\end{itemize}
\end{prop}
This results from [HA, II, \textsection 3, Prop.~2].

(2.6). A map of simplicial rings is said to be a trivial fibration if as a map of
simplicial abelian groups it satisfies the equivalent conditions of 2.5. A map of
simplicial rings is called a cofibration if it has the LLP with respect to all trivial
fibrations of simplicial rings. Cofibrations may be described in the following
alternative way. Call a map $i:R\to S$ of simplicial rings free if there are subsets
$C_q\subset S_q$, $q\geq 0$, such that (i) $\eta^\ast C_q\subset C_p$ whenever
$\eta:[p]\to[q]$ is a surjective monotone map and (ii) $S_q$ is a free $R_q$-algebra
(polynomial ring) with generators $C_q$. Then Theorem 6.1 of [?] implies:

\begin{prop}{2.7}
Any free map of simplicial rings is a cofibration.
\end{prop}
The proof of Propostion 3 [HA, II, \textsection 4] yields

\begin{prop}{2.8}
Any map $f$ of simplicial rings may be factored $f=pi$ where $i$ is free and
$p$ is a trivial fibration.
\end{prop}

\begin{prop}{2.9}
A map $i:R\to S$ of simplicial rings is a cofibration if and only if there is a free map
$j:R\to T$ and maps $u:T\to S$, $v:S\to T$ of simplicial rings under $R$ such that
$uv=\id_S$.
\end{prop}
\begin{proof}
The sufficiency is clear. If $i$ is a cofibration let $i=uj$ be a factorization as
in 2.8. Then $v$ is obtained as the dotted arrow in
\[
  \begin{tikzcd}
    R\ar[r,"j"]\ar[d,"i"'] & T\ar[d,"u"]\\
    S\ar[r,"\id_S"]\ar[ur,dotted] & S.
  \end{tikzcd}
\]
\end{proof}

(2.10). If $K$ is a simplicial set and $Y$ is a simplicial ring, then the function complex
$Y^K$ has a natural structure as a simplcial ring. Let $I=\Delta(1)$ and let
$j_e:Y^I\to Y$, $e=0,1$ (resp. $\sigma:Y\to Y^I$) be the map induced by the inclusion of
the $e$-th vertex $\Delta(0)\to\Delta(1)$ (resp. the unique map $\Delta(1)\to\Delta(0)$).
If $f,g:X\rightrightarrows Y$ are two maps of simplicial rings then a (simplicial)
homotopy from $f$ to $g$ may be identified with a map of simplicial rings $h:X\to Y^I$
such that $j_0 h=f$ and $j_1 h=g$. $\sigma:Y\to Y^I$ is of course the ``constant'' homotopy
from $\id_Y$ to $\id_Y$.

\begin{prop}{2.11}
Let $i:R\to S$ be a cofibration and $f:X\to Y$ a trivial fibration of simplicial rings. Let
$h:R\to X^I$ and $k:S\to Y^I$ be homotopies such that $ki=f^I h$, and let
$\theta_0,\theta_1:S\rightrightarrows X$ be maps with $\theta_e i=j_e h$ and $f\theta_e=j_e k$,
$e=0,1$. Then there is a homotopy $H:S\to X^I$ with $Hi=h$, $f^I H=k$, and $j_e H=\theta_e$,
$e=0,1$.
\end{prop}
\begin{proof}
Consider the square
\[
  \begin{tikzcd}
    R\ar[rr,"h"]\ar[d,"i"'] & & X^I\ar[d,"{(f^I,j_0,j_1)}"]\\
    S\ar[rr,"{(k,\theta_0,\theta_1)}"]\ar[urr,"H",dotted] & & {Y^I\times_{Y^{\dot{I}}}X^{\dot{I}}}
  \end{tikzcd}
\]
where $X^{\dot{I}}=X\times X$. The map $(f^I,j_0,j_1)$ is seen to be a trivial fibration using
2.5(iii), hence the dotted arrow exists and gives te desired homotopy.
\end{proof}

(2.12). Let $u:R\to S$ be a map of simplicial rings. By a cofibrant factorization of $u$ we mean
a factorization $R\xrightarrow{i}T\xrightarrow{p}S$ of $u$ where $i$ is a cofibration and $p$ is
a trivial fibration. If $R\to T'\to S$ is another cofibrant factorization of $u$, then by the
definition of cofibration there are maps $\varphi:T\to T'$ and $\psi:T'\to T$ in the category
$R\backslash\sR/S$. By 2.11 $\varphi\psi$ and $\psi\varphi$ are homotopic to $\id_{T'}$ and
$\id_T$ respectively in this category. Therefore a cofibrant factorization of a map is unique
up to simplicial homotopy under the source and over the target of the map.

(2.13). If $X$ is an object of a category we let $cX$ denote the ``constant'' simplicial object
with $(cX)_q=X$, $\varphi^\ast=\id_X$ for all $q$, $\varphi$.

(2.14). Let $A$, $B$, $M$, and $\C$ be as in \textsection 1. A simplicial $A$-algebra over $B$
(i.e. simplicial object of $\C/B$) $P$ is the same as a factorization $cA\to P\to cB$ of the map
$cA\to cB$. We shall call $P$ a projective $A$-algebra resolution of $B$ if this factorization is
a cofibrant factorization. By 2.8 projective $A$-algebra resolutions of $B$ exist. Choose one $P$,
and let $\LD_{B/A}$ be the simplicial $B$-module $\Diff_{P/A}\otimes_P B$ obtained by applying the
``abelianization'' functor $Y\mapsto\Diff_{Y/A}\otimes_Y B$ dimension-wise to $P$. The simplicial
homotopy type of $\LD_{B/A}$ is independent of the choice of $P$ by 2.12. The underlying chain
complex of $\LD_{B/A}$ is called the cotangent complex of the $A$-algebra $B$ and is unique up
to chain homotopy equivalence of the choice of $P$. The following groups are then well-defined:
\[
  \D_q(B/A,M)=\H_q(\LD_{B/A}\otimes_B M),\tag{2.15}
\]
\[
  \D^q(B/A,M)=\H^q(\Hom_B(\LD_{B/A},M)).\tag{2.16}
\]
We call these the $q$-th homology and cohomology of the $A$-algebra $B$ with values in $M$. When
$M=B$ we write simply $\D_q(B/A)$ and $\D^q(B/A)$.

\begin{prop}{2.14}
The definitions 2.2 and 2.16 are consistent.
\end{prop}
\begin{proof}
We have
\[
  \D^q(B/A,M)=\H^q(\Hom_B(\Diff_{P/A}\otimes_P B,M))=\H^q(\Der(P/A,M)),
\]
so the result follows from [HA, II, \textsection 5, th.~5]. Alternatively it follows from a general
theorem of Verdier (SGA, 1963--1964, Expose~5, Appendix) once one notes that a $\T$-hypercovering of
$B$ is the same as a simplicial $A$-algebra resolution and that a projective resolution is by (2.14)
a cofinal object in the category of $\T$-hypercoverings of $B$.
\end{proof}

\subsection*{\textsection 3. Elementary properties}

In this section we establish properties of the homology and cohomology groups which are elementary in
the sense that they do not use special properties of commutative rings and hence are true without
essential modification for all kinds of universal algebras.

\begin{prop}{3.1}
If $B$ is a free $A$-algebra, then $\LD_{B/A}\simeq c\Diff_{B/A}$ and hence
\[
  \D^q(B/A,M)=\D_q(B/A,M)=0,\quad q>0,
\]
\[
  \D^0(B/A,M)=\Der(B/A,M),
\]
\[
  \D_0(B/A,M)=\Diff_{B/A}\otimes_B M.
\]
\end{prop}
\begin{proof}
The identity map $cB\to cB$ makes $cB$ a free simplicial $A$-algebra resolution of $B$, hence
$\LD_{B/A}\simeq c\Diff_{B/A}$ and the proposition follows.
\end{proof}

(3.2). In analogy with 2.6 we say that a simplicial module $X$ over a simplicial ring $R$ is
free if there are subsets $C_q\subset X_q$ such that (i) $\eta^\ast C_q\subset C_p$ if
$\eta:[p]\to[q]$ is a surjective monotone map and (ii) $X_q$ is a free $R_q$-module with
base $C_q$. $X$ will be called projective if it is a direct summand of a free simplicial
$R$-module.

If $P$ is a free $A$-algebra resolution of $B$ with generators $C_q\subset P_q$ as in 2.6,
then $\{dx\otimes 1 : x\in C_q\}\subset\Diff_{P_q/A}\otimes_{P_q}B$ is a set of generators as
in 3.2. Hence $\Diff_{P/A}\otimes_P B$ is a free simplicial $B$-module. If $Q$ is a projective
$A$-algebra resolution, then by 2.9 $Q$ is a retract of a free one $P$ and so $\Diff_{Q/A}\otimes_Q B$
is a projective simplicial $B$-module. In particular as a chain complex $\Diff_{Q/A}\otimes_Q B$ is
projective in each dimension. Thus

\begin{prop}{3.3}
$\LD_{B/A}$ is a projective simplicial $B$-module.
\end{prop}

\begin{nenv}{Corollary}
If $0\to M'\to M\to M''\to 0$ is an exact sequence of $B$-modules then there are long exact
sequences
\[
  0\lra\D^0(B/A,M')\lra\D^0(B/A,M)\lra\D^0(B/A,M'')\lra\D^1(B/A,M')\lra\cdots,\tag{3.4}
\]
\[
  \cdots\lra\D_1(B/A,M'')\lra\D_0(B/A,M')\lra\D_0(B/A,M)\lra\D_0(B/A,M'')\lra 0.\tag{3.5}
\]
\end{nenv}

\begin{nenv}{Corollary}
There are universal coefficient spectral sequences
\[
  E_{pq}^2=\Tor_p^B(\D_q(B/A),M)\Lra\D_{p+q}(B/A,M),\tag{3.6}
\]
\[
  E_2^{pq}=\Ext_B^p(\D_q(B/A),M)\Lra\D^{p+q}(B/A,M).\tag{3.7}
\]
\end{nenv}

\begin{prop}{3.8}
$\D^0(B/A,M)=\Der(B/A,M)$, $\D_0(B/A,M)=\Diff_{B/A}\otimes_B M$.
\end{prop}
\begin{proof}
As $Y\mapsto\Diff_{Y/A}\otimes_Y B$ is a left adjoint functor by 1.6, it is
right exact. Hence $\D_0(B/A)=\Diff_{B/A}$ and the proposition follows from 3.6 and 3.7.
\end{proof}

\begin{envr}{Remark}{3.9}
The spectral sequence 3.7 shows that $M\mapsto\D^q(B/A,M)$ is the $q$-th derived functor of
$M\mapsto\Der(B/A,M)$ if and only if $\D_q(B/A)=0$ for $q>0$. This can fail to be so even if
$A$ is a field, and in this respect commutative ring cohomology is quite distinct from group,
Lie algebra, and associative algebra cohomology.
\end{envr}

(3.10). By an extension of the $A$-algebra $B$ by $M$ we mean an exact sequence
\[
  0\lra M\xrightarrow{\ i\ }X\xrightarrow{\ u\ }B\lra 0\tag{3.11}
\]
where $u$ is a map of $A$-algebras such that $(\ker u)^2=0$ and where $i$ induces an
isomorphism of $B$-modules $M\xrightarrow{\sim}\ker u$, $\ker u$ being endowed with the
$B$-module structure $u(x)y=x\cdot y$, $x\in X$, $y\in\ker u$. Let $\Exalcomm(B/A,M)$
be the set of isomorphism classes of extentions of the $A$-algebra $B$ by $M$.

\begin{prop}{3.12}
There is a canonical bijection
\[
  \D^1(B/A,M)\simeq\Exalcomm(B/A,M).
\]
\end{prop}
\begin{proof}
Given a free $A$-algebra resolution $P$ of $B$ and an extention 3.11, choose a map
$\theta:P_0\to X$ of $A$-algebras over $B$. Then $i^{-1}\{\theta d_0-\theta d_1\}:P_1\to M$
is a normalized $1$-cocycle of $\Der(P/A,M)$ whose cohomology class is independent of the
choice of $\theta$. This gives a map $\Phi:\Exalcomm(B/A,M)\to\D^1(B/A,M)$ which one may
show is independent of the choice of $P$. Conversely given an $A$-derivation $D:P_1\to M$ let
\[
  \begin{tikzcd}
    {X=\coker\big(P_1}\ar[rr,"{(d_0,D)}",shift left]\ar[rr,"{(d_1,0)}"',shift right] & & {P_0\oplus M\big),}
  \end{tikzcd}
\]
the cokernel being in the category of $A$-modules, and let $p:P_0\oplus M\to X$ be the canonical
projection. Then $X$ is a quotient $A$-algebra of $P_0\oplus M$ and if $D$ is a cocycle of
$\Der(P/A,M)$ we obtain an extention 3.11 with $u(p(y\oplus m))=\varepsilon y$, $i(m)=p(0\oplus m)$.
It is staightforward to verify that this procedure gives an inverse to $\Phi$.
\end{proof}

\begin{envr}{Remark}{3.13}
It is also possible to prove 3.12 using the interpretation of the sheaf cohomology in terms of
``torsors.'' By [?] one also gets a general interpretation of $\D^2$ this way, however for rings
the methods of the second chapter seem better for handling the higher cohomology.
\end{envr}

\begin{env}{Corollary}{3.14}
Suppose that $A/I\simeq B$ where $I$ is an ideal in $A$. Then $\D_0(B/A)=0$
and $\D_1(B/A)\simeq I/I^2$.
\end{env}
\begin{proof}
$\D_0(B/A)=0$ by 3.8. From 3.7 and 3.12 we have $\Hom_B(\D_1(B/A),M)\simeq\D^1(B/A,M)\simeq\Exalcomm(B/A,M)$.

Let $\chi\in\Exalcomm(B/A,I/I^2)$ be the isomorphism class of the extention $0\to I/I^2\to A/I^2\to B\to 0$.
By functorality $\chi$ defines a natural transformation $\chi_\ast:\Hom_N(I/I^2,M)\to\Exalcomm(B/A,M)$.
Conversely given an extention 3.11, choose a map $\theta:A\to X$ with $u\theta=i_B$; the restriction of
$\theta$ to $I$ gives rise to a homomorphism of $B$-modules $I/I^2\to M$. This procedure is easily seen to
define an inverse to $\chi_\ast$, hence $\Hom_B(\D_1(B/A),M)\simeq\Hom_B(I/I^2,M)$ and so $\D_1(B/A)\simeq I/I^2$.
\end{proof}

\subsection*{\textsection 4. Further properties}

In this section we establish properties of the cohomology theory which are peculiar to commutative rings, since
they depend on the fact that the direct sum in the category of $A$-algebras is given by the tensor product. We
begin by recalling facts about simplicial modules which are proved in [HA, II, \textsection6].

(4.1). Let $R$ be a simplicial ring and let $\M_R$ be the abelian category of simplicial $R$-modules. The
homotopy category $\Ho(\M_R)$ is obtained from $\M_R$ by formally adjoining the inverses of the weak equivalences
(maps which induce isomorphisms on homology). $\Ho(\M_R)$ is equivalent to the category whose objects are
the projective simplicial $R$-modules (3.2) with homotopy classes of maps for morphisms. When $R=cA$, 
$\Ho(\M_R)$ is equivalent to the full subcategory of the derived category of $A$-modules consisting of the
chain complexes.

(4.2). If $X$, $Y$ are two simplicial $R$-modules, let $\Tor_i^R(X,Y)$ be the simplicial $R$-module
obtained by first applying the tri-functor $\Tor_i^-(-,-)$ dimension-wise. Let $X\otimes_R^\L Y$ denote
the derived tensor product of $X$ and $Y$. $X\otimes_R^\L Y$ is isomorphic in $\Ho(\M_R)$ to
$P\otimes_R Q$, where $P$ and $Q$ are projective resolutions of $X$ and $Y$ respectively. There is a
spectral sequence ([HA, II, \textsection 6, th.~6])
\[
  E_{pq}^2=\H_p(\Tor_q^R(X,Y))\Lra\H_{p+q}(X\otimes_R^\L Y)\tag{4.3}
\]
one of whose edge homomorphisms is the map on homology induced by the canonical map
$X\otimes_R^\L Y\simeq P\otimes_R Q\to X\otimes_R Y$. Consequently we have

\begin{prop}{4.3}
If $\Tor_q^R(X,Y)=0$ for $q>0$, then $X\otimes_R^\L Y\simeq X\otimes_R Y$.
\end{prop}

(4.4). Let $u:R\to S$ be a map of simplicial rings. Define $\LD_{S/R}$ to be the projective
simplicial $S$-module $\Diff_{P/R}\otimes_P S$ where $R\to P\to S$ is a cofibrant factorization of $u$
(2.12). As an object of $\Ho(\M_S)$ it is independent up to isomorphism of the choice of factorization.
If $X$ is a simplicial $S$-module we define
\[
  \D_q(S/R,X)=\H_q(\LD_{S/R}\otimes_S^\L X)\simeq\H_q(\LD_{S/R}\otimes_S X),\tag{4.5}
\]
where the last isomorphism is from 4.3. It is clear that this definition specializes to 2.15
in the case where $u$ is the map $cA\to cB$ and $X=cM$. Moreover the obvious generalization of 3.5
holds and 3.6 generalizes by ([HA, II, th.~6(c)]) to a spectral sequence
\[
  E_{pq}^2=\H_p(\D_q(S/R)\otimes_S^\L X)\Lra\D_{p+q}(S/R,X),\tag{4.6}
\]
where $\D_q(S/R)=\D_q(S/R,S)$.

\begin{nenv}{Proposition}
Suppose $u:R\to R'$ and $v:R\to S$ are maps of simplicial rings such that $\Tor_q^R(R',S)=0$ for
$q>0$. If $S'=S\otimes_R R'$, then there are canonical isomorphisms in $\Ho(\M_{S'})$
\[
  \LD_{S/R}\otimes_R R'\simeq\LD_{S'/R'},\tag{4.7}
\]
\[
  \LD_{S'/R}\simeq\LD_{S/R}\otimes_R R'\oplus\LD_{R'/R}\otimes_R S.\tag{4.8}
\]
\end{nenv}
\begin{proof}
First observe that if $R\to P$ is a cofibration (resp. free map) of simplicial rings, then $P$ is
a projective (resp. free) $R$-module. Using 2.9, one reduces to the case where $P$ is free, whence
if $C_\ast$ is an $R$-algebra basis for $P$ as in 2.6, then the monomials in the elements of $C_\ast$
form an $R$-module basis for $P$ as in 3.2.

Now let $R\to P\to S$ be a cofibrant factorization of $R\to S$. As $P\to S$ is a weak equivalence of
$R$-modules, so is $P\otimes_R^\L R'\to S\otimes_R^\L R'$, since $-\otimes_R^\L-$ is a functor on
$\Ho(\M_R)$. By hypothesis, the fact that $P$ is a projective $R$-module by the above remarks, and
4.3, this map is isomorphic to the map $P\otimes_R R'\to S'$. Hence this last map is a weak equivalence;
as it is a trivial fibration. As cobase extension preserves cofibrations, it follows that
$R'\to P\otimes_R R'\to S'$ is a cofibrant factorization of $R'\to S'$ and hence setting $P'=P\otimes_R R'$
\[
  \LD_{S'/R'}=\Diff_{P'/R'}\otimes_{R'}S'\simeq(\Diff_{P/R}\otimes_R S)\otimes_R R'\simeq\LD_{S/R}\otimes_R R'
\]
which proves 4.7. To prove 4.8, let $R\to Q\to R'$ be a cofibrant factorization of $R\to R'$. By the same
argument as above we find that $R\to P\otimes_R Q\to S'$ is a cofibrant factorization of $R\to S'$ and so
\begin{align*}
  \LD_{S'/R'}&=\Diff_{P\otimes_R Q/R}\otimes_{P\otimes_R Q}S'\simeq(\Diff_{P/R}\otimes_R Q\oplus\Diff_{Q/R}\otimes_R P)\otimes_{P\otimes_R Q}S'\\
              &\simeq(\Diff_{P/R}\otimes_P S)\otimes_R R'\oplus(\Diff_{Q/R}\otimes_Q R')\otimes_R S\\
              &\simeq\LD_{S/R}\otimes_R R'\oplus\LD_{R'/R}\otimes_R S.
\end{align*}
The resulting isomorphisms in $\Ho(\M_{S'})$ are canonical since they are independent of the choices of $P$,
$Q$ by 2.11.
\end{proof}

\begin{env}{Corollary}{4.9}
Let $B$ and $C$ be $A$-algebras and let $N$ be a $B\otimes_A C$-module. If $\Tor_q^A(B,C)=0$ for $q>0$, then
there are isomorphisms
\[
  \D^q(B\otimes_A C/C,N)=\D^q(B/A,N),
\]
\[
  \D^q(B\otimes_A C/A,N)=\D^q(B/A,N)\oplus\D^q(C/A,M),
\]
and similarly for homology.
\end{env}

(4.10). If $0\to X'\to X\to X''\to 0$ is an exact sequence in $\M_R$, then in $\Ho(\M_R)$ there is a cofibration
sequence ([HA, II, \textsection 6])
\[
  X'\lra X\lra X''\xrightarrow{\ \partial\ }\Sigma X',\tag{4.11}
\]
where $\Sigma$ denotes the suspension functor on $\Ho(\M_R)$. When $R=cB$ and we identify $\Ho(\M_R)$ with
the subcategory of the derived category of $B$-modules consisting of chain complexes, then the suspension functor
shifts a complex to the left and the sequence 4.11 is the distinguished triangle associated to the exact sequence.

\begin{nenv}{Theorem}
Let $R\xrightarrow{u}S\xrightarrow{v}T$ be maps of simplicial rings. Then there is a canonical cofibration sequence
$\Ho(\M_T)$
\[
  \LD_{S/R}\otimes_S T\lra\LD_{T/R}\lra\LD_{T/S}\lra\Sigma(\LD_{S/R}\otimes_S T).\tag{4.12}
\]
\end{nenv}
\begin{proof}
Form a diagram
\[
  \begin{tikzcd}
    & & Q\ar[d,"i_2"]\ar[rdd,"q"]\\
    & P\ar[ur,"j"]\ar[d,"p"] & S\otimes_P Q\ar[rd,"r"]\\
    R\ar[ur,"i"]\ar[r,"u"] & S\ar[ur,"i_1"]\ar[rr,"v"] & & T
  \end{tikzcd}
  \tag{4.13}
\]
by choosing cofibrant factorizations $u=pi$ and $vp=qj$, and then filling in the rest of the diagram in the obvious
way.

As $j$ is a cofibration $Q$ is a projective $P$-module, hence by 4.3 the map $i_2$ is isomorphic to
$p\otimes_P^\L\id_Q:P\otimes_P^\L Q\to S\otimes_P^\L Q$, which is a weak equivalence since $p$ is. As $i_2$ and
$q$ are weak equivalences so is $r$; $r$ is also surjective since $q$ is and therefore $r$ is a trivial fibration
of simplicial $S$-algebras. As $j$ is a cofibration so is $i_1$; therefore $v=ri_1$ is a cofibrant factorization.

Suppose for the moment that 4.13 is a diagram of rings where $Q$ is a free $P$-algebra. If $N$ is a $T$-module,
then there is an exact sequence
\[
  0\lra\Der(Q/P,N)\lra\Der(Q/R,N)\xrightarrow{\ j^\ast\ }\Der(P/R,N)\lra 0
\]
where $j$ is onto because $Q$ is a free $P$-algebra and hence there is an $R$-algebra map from $Q$ to $P$ which
is left inverse to $j$. As this sequence is functorial in $N$ it comes from an exact sequence of $T$-modules
\[
  0\lra(\Diff_{P/R}\otimes_P S)\otimes_S T\lra\Diff_{Q/R}\otimes_Q T\lra\Diff_{S\otimes_P Q/S}\otimes_{S\otimes_P Q}T\lra 0.\tag{4.14}
\]
This last sequence is seen to be functorial in the diagram of rings 4.13, hence applying this functorial exact sequence
dimension-wise to 4.13 now considered as a diagram of simplicial rings we obtain an exact sequence 4.14 of simplicial
$T$-modules.

Now using that $S\otimes_P Q$ is a projective $S$-algebra resolution of $T$, that $Q$ is a projective $R$-algebra resolution
of $T$, and that $P$ is a projective $R$-algebra resolution of $S$, we see that the cofibration sequence in $\Ho(\M_T)$
associated to 4.14 is the desired cofibration sequence 4.12.

It remains to show this cofibration sequence is independent of the choice of diagram 4.13. Suppose given a diagram of
simplicial rings
\[
  \begin{tikzcd}
    R\ar[r]\ar[d] & S\ar[r]\ar[d] & T\ar[d]\\
    R'\ar[r] & S'\ar[r] & T'
  \end{tikzcd}
  \tag{4.15}
\]
and suppose given a diagram (4.13)$'$ similar to 4.13 but with primes. By lifting successively in the diagrams
\[
  \begin{tikzcd}
    R\ar[r]\ar[d] & P'\ar[d]\\
    P\ar[ur,dotted]\ar[r] & S'
  \end{tikzcd}
  \quad
  \begin{tikzcd}
    P\ar[r]\ar[d] & Q'\ar[d]\\
    Q\ar[ur,dotted]\ar[r] & T'
  \end{tikzcd}
  \tag{4.16}
\]
we obtain a map from 4.13 to (4.13)$'$, hence a map from the exact sequence 4.14 to the corresponding exact sequence
(4.14)$'$, and finally a map of cofibration sequences. The resulting map is independent of the choices of liftings in
4.16 because by 2.11 two liftings in the first square are joined by a homotopy which may then be extended to a homotopy
between the liftings in the second square. It follows that the map of exact sequences 4.14 to (4.14)$'$ is unique up to
homotopy and hence the map of cofibration sequences is well-defined.
\end{proof}

Specializing to constant simplicial rings, we have
\begin{env}{Corollary}{4.17}
If $A\to B\to C$ are maps of rings, then there is a canonical exact triangle in the derived category of $C$-modules
\[
  \begin{tikzcd}
    \LD_{B/A}\otimes_B C\ar[rr] & & \LD_{C/A}\ar[ld]\\
    & \LD_{C/B}\ar[ul,dotted].
  \end{tikzcd}
\]
Hence if $M$ is a $C$-module, there are canonical exact sequences
\[
  0\lra\D^0(C/B,M)\lra\D^0(C/A,M)\lra\D^0(B/A,M)\lra\D^1(C/B,M)\lra\cdots,
\]
\[
  \cdots\lra\D_1(C/B,M)\lra\D_0(B/A,M)\lra\D_0(C/A,M)\lra\D_0(C/B,M)\lra 0.
\]
\end{env}

\subsection*{\textsection 5. Some applications}

In this section we extend to all $q$ certain vanishing results for $\D^q(B/A,M)$ which we proved in [?] for
$q=1$ and in [?] for $q=1,2$. We shall state these results only for the cotangent complex $\LD_{B/A}$ leaving
the translation for the functors $\D^q$ and $\D_q$ to the reader.

\begin{prop}{5.1}
If $S$ is a multiplicative system in $A$, then $\LD_{S^{-1}A/A}\simeq 0$.
\end{prop}
\begin{proof}
(after Andr{\'e} [?, 20.1]) Let $C=S^{-1}A$. As $C\otimes_A C\simeq C$ we have an isomorphism
$\LD_{C/A}\otimes_A C\simeq\LD_{C/A}$ of projective simplicial $C$-modules. As $C$ is flat over
$A$ the former complex by 4.7 is isomorphic to $\L D_{C\otimes_A C/C}\simeq\LD_{C/C}\simeq 0$.
\end{proof}

\begin{env}{Corollary}{5.2}
Suppose $T$ is a multiplicative system in $B$ and that $S$ is a multiplicative system in $A$ which
is carried into $T$ by the homomorphism $A\to B$. Then
\[
  \LD_{T^{-1}B/S^{-1}A}\simeq\LD_{B/A}\otimes_B T^{-1}B.
\]
\end{env}
\begin{proof}
Applying 4.17 to $A\to B\to T^{-1}B$ and $A\to S^{-1}A\to T^{-1}B$ and using 5.1 we have
\[
  \LD_{B/A}\otimes_B T^{-1}B\simeq\LD_{T^{-1}B/A}\simeq\LD_{T^{-1}B/S^{-1}A}.
\]
\end{proof}

\begin{prop}{5.3}
Suppose that $A$ is Noetherian and $B$ is of finite type as an $A$-algebra. Then
\begin{center}
  $B$ is {\'e}tale over $A\iff\LD_{B/A}\simeq 0$;
\end{center}
\begin{center}
  $B$ is smooth over $A\iff\LD_{B/A}\simeq c\Diff_{B/A}$ and $\Diff_{B/A}$ is a projective $B$-module.
\end{center}
\end{prop}
\begin{proof}
($\Lra$) Suppose $B$ {\'e}tale over $A$, i.e. $B$ is flat over $A$ and $\Delta:\Spec B\to\Spec B\otimes_A B$
is an open immersion. Let $\mathfrak{p}$ be a prime ideal of $B$ and let $\mathfrak{q}=\Delta(\mathfrak{p})$
so that $(B\otimes_A B)_{\mathfrak{q}}\simeq B_{\mathfrak{p}}$. Then there is an isomorphism of projective
$B_{\mathfrak{p}}$-modules
\begin{align*}
  (\LD_{B/A})_{\mathfrak{p}}&\simeq(\LD_{B/A}\otimes_A B)_{\mathfrak{q}}\\
                             &\simeq(\LD_{B\otimes_A B/B})_{\mathfrak{q}}\quad\text{by 4.9 since $B$ is flat over $A$.}\\
                             &\simeq\LD_{(B\otimes_A B)_{\mathfrak{q}}/B_{\mathfrak{p}}}\quad\text{by 5.2}\\
                             &\simeq 0.
\end{align*}
Thus $\H_\ast((\LD_{B/A})_{\mathfrak{p}})=\H_\ast(\LD_{B/A})_{\mathfrak{p}}=0$ as as $\mathfrak{p}$ is an arbitrary
prime ideal of $B$, we have $\LD_{B/A}\simeq 0$.

Now suppose $B$ smooth over $A$. As $\H_0(\LD_{B/A})\simeq\Diff_{B/A}$ (3.8) there is a canonical map of
projective simplicial $B$-modules $\LD_{B/A}\to c\Diff_{B/A}$. To prove this is an isomorphism we reduce by
localizing on $B$ to the case where $A\to B$ may be factoried $A\to P\to B$ where $P$ is a polynomial
ring over $A$ and $P\to B$ is {\'e}tale. Then by 4.17, the {\'e}tale case of 5.3, and 3.1 we have
\[
  \begin{tikzcd}
    \LD_{P/A}\otimes_P B\ar[r,"\sim"]\ar[d,"\sim"'] & \LD_{B/A}\ar[d]\\
    \Diff_{P/A}\otimes_P B\ar[r,"\sim"] & c\Diff_{B/A}
  \end{tikzcd}
\]
which proves the assertion.

($\Longleftarrow$) The spectral sequence 3.7 degenerates yielding $\D^1(B/A,M)=0$ for all $B$-modules $M$.
By 3.12 all $A$-algebra extentions of $B$ by an ideal of square zero split hence $B$ is smooth over $A$ by
SGA, 1960--1961, III, 2--3. If also $\Diff_{B/A}=0$, then $B$ is {\'e}tale over $A$.
\end{proof}

We know wish to give a reasonably ``geometric'' example where $\D_1(B/A)\neq 0$. The following results from
4.17 and 3.14.
\begin{prop}{5.4}
Suppose that $A\to P\to B$ is a factorization of $A\to B$ where $P$ is a polynomial ring
over $A$ and $P\to B$ is surjective with kernel $I$. Then
\[
  \D_q(B/A,M)\simeq\D_q(B/P,M),\quad q\geq 2,
\]
and there is an exact sequence
\[
  0\lra\D_1(B/A,M)\lra I/I^2\otimes_B M\lra\Diff_{P/A}\otimes_P M\lra\Diff_{B/A}\otimes_B M\lra 0
\]
and similar assertions hold for cohomology.
\end{prop}

\begin{envr}{Example}{5.5}
Supose that $k$ is an algebraically closed field and that $R$ is the coordinate ring
of the curve $x=t^3$, $y=t^4$, $z=t^5$. Then $R$ is an integral domain finitely generated
over $k$. We show that $\D_1(R/k)\neq 0$. $R=P/I$ where $P=k[X,Y,Z]$ and
$I=(Y^2-XZ,YZ-X^3,Z^2-XY)$. The element $u=XY(Y^2-XZ)-X(YZ-X^3)+Z(Z^2-X^2 Y)$ is in $I$
but not in $I^2$. In effect if $\mathfrak{m}=(X,Y,Z)$, $I^2\subset\mathfrak{m}^4$ and
$u\not\in\mathfrak{m}^4$. But by 5.4 we have the exact sequence
\[
  0\lra\D_1(R/k)\lra I/I^2\xrightarrow{\ \delta\ }\Diff_{P/k}\otimes_P R\lra\Diff_{R/k}\lra 0
\]
and a short calculation shows that $\delta(u+I^2)=0$. Hence $\D_1(R/k)\neq 0$ as
asserted. It may be worth remarking that $I$ is a prime ideal in $P$ such that $I^2$
is not primary; in fact $\im\delta=I/I^{(2)}$ where $I^{(2)}$ is the $I$-primary
component of $I^2$.
\end{envr}

\section*{Chapter~II. The fundamental spectral sequence}

In order to calculate $\D_\ast(B/A,M)$ one is reduced by 4.17 to the case where
$B=A/I$, $I$ an ideal in $A$. In this case there is a spectral sequence which relates
these groups to $\Tor_\ast^A(B,M)$, which is more easily computable. In this chapter
we derive this spectral sequence and give some of its applications.

\subsection*{\textsection 6. Construction of the spectral sequence}

We retain the notations of the preceding chapter except that certain rings will not be
commutative, but skew-commutative with respect to a canonical grading.

Let $P$ be a free simplicial $A$-algebra resolution of $B$. Then $Q=P\otimes_A B$ is a
simplicial augmented $B$-algebra. If $J=\ker(P\otimes_A B\to B)$ is the augmentation
ideal, then
\[
  Q\supset J\supset J^2\supset\cdots\tag{6.1}
\]
is a filtration of $Q$ by simplicial ideals. By means of the shuffle operation
$\underline{\otimes}$ ([HA, II, p.~6.6, (6)]), $Q$ with differential $d=\sum_i(-1)^i d_i$
becomes a skew-commutative differential graded ring and 6.1 is a filtration of $Q$ by
differential graded ideals. Consequently we obtain a spectral sequence of alegbras where
\[
  E_{pq}^2=\H_{p+q}(J^q/J^{q+1}),\quad d^r:E_{pq}^r\lra E_{p-r,q+r-1}^r,\tag{6.2}
\]
and ignoring for the moment questions of convergence, whose abutment is
$\H_\ast(Q)\simeq\Tor_\ast^A(B,B)$. As $P$ is free over $A$, $Q$ is free over $B$ hence
there is an isomorphism of graded simplicial algebras
\[
  \textstyle\bigoplus_q S_q^B(J/J^2)\simeq\bigoplus_q J^q/J^{q+1},\tag{6.3}
\]
where the left side the symmetric algebra functor over $B$ applied dimension-wise to
the simplicial $B$-module $J/J^2$. Applying 5.4 dimension-wise to the maps
$cA\to P\to cB$ we obtain isomorphisms of simplicial $B$-modules
\[
  J/J^2\simeq\Diff_{P/A}\otimes_P B\simeq\LD_{B/A}.\tag{6.4}
\]
We now turn to the convergence of this spectral sequence.

\begin{env}{Lemma}{6.5}
Suppose that $Q$ is a projective augmented simplicial $B$-algebra with augmentation
idela $J$. If $\H_0(J)=0$, then $\H_k(J^n)=0$ for $k<n$. 
\end{env}

A more general proof of this lemma will be proven in 8.8. An alternative proof of 6.5
in outline is as follows. The arguments in [?, \textsection 4] are very general and show
that it is sufficient to prove 6.5 when $Q=S^B X$ and $X=K(B,1)^r$ where
$K(B,1)=B\Delta(1)/B\Delta(1)$ is the simplicial $B$-module whose normalization is
the complex with $B$ in dimension $1$ and $0$ elsewhere. In this case one may apply known
results on the connectivity of the symmetric algebra functor [?], in particular the
following which will be proved in 7.32.

\begin{nenv}{Lemma}
Suppose that $X$ is a flat simplicial $B$-module with $\H_0(X)=0$. Then
\[
  \H_q(S_n^B X)=0,\quad q<n,
\]
and there is a graded algebra isomorphism
\[
  \textstyle\bigoplus_n\wedge_n^B\H_1(X)\simeq\bigoplus_n\H_n(S_n^B X),
\]
where $\wedge^B$ is the exterior algebra functor on $B$-modules.
\end{nenv}

In virtue of the augmentation, $\H_0(J)=0$ is equivalent to $\H_0(Q)\simeq B$,
which when $Q=P\otimes_A B$ means $B\otimes_A B\simeq B$. In this case the spectral
sequence 6.2 consructed from the $J$-adic filtration on $Q$ converges by 6.5, that is,
$E_{pq}^r=E_{pq}^\infty$ for $r>p+q$ and moreover $E_{pq}^2=0$ if $p$ of $q<0$ by 6.5.
Combining 6.2--6.4, we therefore have

\begin{env}{Theorem}{6.8}
If $B\otimes_A B\simeq B$, then there is a first quadrant spectral sequence
\[
  E_{pq}^2=\H_{p+q}(S_q^B\LD_{B/A})\Lra\Tor_{p+q}^A(B,B)
\]
of bigraded algebras, skew-commutative for the total degree.
\end{env}

\noindent
Picture of spectral sequence:
\[
  \begin{sseq}[entrysize=9mm,
               xlabelstep=3,ylabelstep=4,
               xlabels={;p},ylabels={;q}]{0...3}{0...4}
   \ssdrop{B}
   \ssmoveto 0 1
    \ssdrop{\D_1}
    \ssmoveto 0 2
    \ssdrop{\wedge^2\D_1}
    \ssmoveto 0 3
    \ssdrop{\wedge^3\D_1}
    \ssmoveto 1 0
    \ssdrop{0}
    \ssmoveto 2 0
    \ssdrop{0}
    \ssmoveto 1 1
    \ssdrop{\D_2}
    \ssmoveto 2 1
    \ssdrop{\D_3}
  \end{sseq}.
\]

\noindent
Edge homomorphisms
\[
  \Tor_n^A(B,B)\lra\D_n(B/A),\quad n>0,\tag{6.9}
\]
\[
  \wedge_n^B\D_1(B/A)\lra\Tor_n^A(B,B).\tag{6.10}
\]

\noindent
Low dimensional isomorphisms
\[
  \D_0(B/A)=0,
\]
\[
  \D_1(B/A)\simeq\Tor_1^A(B,B).\tag{6.11}
\]

\noindent
Five-term exact sequence
\[
  \Tor_3^A(B,B)\lra\D_3(B/A)\xrightarrow{\ d_2\ }\wedge_2^B\D_1(B/A)\lra
  \Tor_2^A(B,B)\lra\D_2(B/A)\lra 0.\tag{6.12}
\]
6.10 is the unique graded $B$-algebra morphism extending the isomorphism 6.11.

When $B=A/I$ where $I$ is an ideal in $A$ the hypothesis of 6.8 holds and in this case
we may avail ourselves of the isomorphism
\[
  \D_1(B/A)\simeq\Tor_1^A(B,B)\simeq I/I^2,\tag{6.13}
\]
and rewrite the edge homomorphism 6.10
in the form
\[
  \wedge_n^B(I/I^2)\lra\Tor_n^A(B,B).\tag{6.14}
\]

\begin{prop}{6.15}
The edge homomorphism (6.9) annihilates the decomposable elements of $\Tor_\ast^A(B,B)$.
\end{prop}
\begin{proof}
For $n>0$, $\Tor_n^A(B,B)=\H_n(Q)\simeq\H_n(J)$. If $\alpha\in\Tor_p$, $p>0$, is
represented by $x\in J_p$ and $\beta\in\Tor_q$, $q>0$, is represented by $y\in J_q$,
then $\alpha\cdot\beta\in\Tor_{p+q}$ is represented by
$\mu(x\underline{\otimes}y)\in J_{p+q}^2$, where $\mu:Q\otimes Q\to Q$ is the
multiplication. But the edge homomorphism 6.9 is induced by $J\to J/J^2$, hence the
image of $\alpha\cdot\beta$ in $\D_{p+q}(B/A)$ is zero.
\end{proof}

If $M$ is a $B$-module the as $J^q$ and $J/J^2$ are projective simplicial $B$-modules
\[
  Q\otimes_B M\supset J\otimes_B M\supset\cdots
\]
is a filtered simplicial module over the filtered simplicial ring $Q$ with
\[
  \gr Q\otimes_B M\simeq S^B(J/J^2)\otimes_B M.
\]
Hence
\begin{env}{Theorem}{6.16}
If $B\otimes_A B\simeq B$, then there is a spectral sequence
\[
  E_{pq}^2=\H_{p+q}(S_q^B\LD_{B/A}\otimes_B M)\Lra\Tor_{p+q}^A(B,M)
\]
which is a spectral sequence of modules over the spectral sequence 6.12.
\end{env}

It is easy to verify that this spectral sequence has the following properties:

Edge homomorphisms:
\[
  \Tor_n^A(B,M)\lra\D_n(B/A,M),\quad n>0,
\]
\[
  \wedge_n^B\D_1(B/A)\otimes_B M\lra\Tor_n^A(B,M).
\]

Low-dimensional isomorphisms:
\[
  \D_0(B/A,M)=0,
\]
\[
  \D_1(B/A,M)=\Tor_1^A(B.M)\simeq\D_1(B/A)\otimes_B M.
\]

Five-term exact sequence:
\[
  \Tor_3^A(B,M)\lra\D_3(B/A,M)\lra\wedge^2\D_1(B/A)\otimes_B M\lra\Tor_2^A(B,M)\lra
  \D_2(B/A,M)\lra 0
\]
and that $\Tor_+^A(B,B)\Tor_+^A(B,M)\subset\Tor_+^A(B,M)$ is annihilated by the edge
homomorphism.

\begin{envr}{Remark}{6.17}
The condition $B\otimes_A B\simeq B$ is necessary as is shown by the Example 5.5
where $B$ is flat over $A$ and $\D_1(B/A)\neq 0$.
\end{envr}

\begin{envr}{Remark}{6.18}
The spectral sequences 6.8 and 6.16 are functorial in the triple $A$, $B$, $M$ since
the only choice made in their construction was the free $A$-algebra resolution $P$ of
$B$ which is seen to be unique and functorial in $A$, $B$ using 2.11.
\end{envr}

\subsection*{\textsection 7. Homology of the symmetric algebra}

In order to use the spectral sequence 6.8 it is necessary to have results relating to the
homology of the symmetric algebra of a simplicial module with the homology of the module.
In this rather long section we collect the results that we need. They include a
connectivity assertion (7.3), calculation of the first non-vanishing homology groups
(7.27), and a calculation in the case where the ground ring is of characteristic zero
(7.43). The symmetric algebra functor is closely connected with Eilenberg-MacLane spaces
in topology and at tthe end of this section we outline this connection.

(7.1). Let $F$ be a functor defined on the category of ring-modules $(B,M)$ consisting
of a ring $B$ and a $B$-module $M$ and having values in an abelian category $\A$.
The functors we have in mind are the symmetric algebra $S$, the exterior algebra
$\wedge$, the divided power algebra $\Gamma$, as well as any of the tensor products built
up from homogeneous components of these functors. If $R$ is a simplicial ring and
$X$ is a simplicial $R$-module, then applying $F$ dimension-wise to $X$ we obtain a
simplicial object $F(R,X)$. For the most part $R$ will be fixed and we will write
simply $F(X)$ when there is no possibility of confusion.

The left-derived functor $\L F$ of $F$ is defined by $\L F(X)=F(P)$, where
$P\to X$ is a projective resolution of $X$. The homotopy type of the simplicial object
$\L F(X)$ is independent of the choice of $P$ and $\L F$ is a functor from $\Ho(\M_R)$
to the category $\pi_0\,\sA$ whose objects are the same as $\sA$ but with homotopy
classes of maps for morphisms. The map $P\to X$ gives rise to a natural transformation
$\L F(X)\to F(X)$.

(7.2). A map $f:X\to Y$ of simplicial objects in an abelian category will be called
a $k$-equivalence if $f_\ast:\H_q(X)\to\H_q(Y)$ is an isomorphism for $q\leq k$. $X$
is said to be $k$-connected if $\H_q(X)=0$ for $q<k$.

\begin{prop}{7.3}
If $f:X\to Y$ is a $k$-equivalence so is $\L F(f)$.
\end{prop}
\begin{proof}
We may assume $X$ and $Y$ are free simplicial $R$-modules and drop the $\L$. By
``attaching cells'' to $X$ we will now construct a free map $X\to X'$ which is an
isomorphism in dimensions $\leq k+1$ such that $\H_q(X')=0$ for $q>k$. If $X$ is a
simplicial set, let $RK$ be the free simplicial $R$-module generated by $K$
($K\mapsto RK$ is left adjoint to the forgetful functor
$\M_R\to\mathrm{s}\Set$). Let $\alpha_i\in\H_{k+1}(X)$ be elements
which generate $\H_{k+1}(X)$ as an $\H_0(X)$-module and choose a representative
$x_i\in N_{k+1}(X)$ for $\alpha_i$. Here $N(X)$ is the normalized chain complex of $X$.
Let $u_i:R\Delta(k+2)\to X$ be the unique simplicial $R$-module map sending
$d_j\id_{[k+2]}$ to $x_i$ for $j=0$ and $0$ for $j=1,\dots,k+2$. Define
$X\to X^{(1)}$ by a co-cartesian diagram
\[
  \begin{tikzcd}
    {\bigoplus_i R\Delta(k+2)}\ar[r]\ar[d,"\sum_i u_i"']
    & {\bigoplus_i R\Delta(k+2)}\ar[d]\\
    X\ar[r] & X^{(1)}.
  \end{tikzcd}
\]
The map $X\to X^{(1)}$ is an isomorphism in dimensions $\leq k+1$. The cokernel
of both horizontal maps of this square is
\[
  \textstyle\bigoplus_i R\Delta(k+2)/R\Delta(k+2)
\]
whose homology is a free $\H_\ast R$-module on generators of dimension $k+2$
corresponding to the elements of $I$ (see [HA, II, p.~5.11, assertion~A]). The
long exact sequence in homology for the exact sequence containing $X\to X^{(1)}$
is thus
\[
  \cdots\lra\textstyle\bigoplus_I\H_0(R)\xrightarrow{\ \delta\ }\H_{k+1}(X)
  \lra\H_{k+1}(X^{(1)})\lra 0\lra\cdots.
\]
By contruction $\delta$ is surjective hence $\H_{k+1}(X^{(1)})=0$. Repeating this
contruction we obtain free maps $X^{(n)}\to X^{(n+1)}$ which are isomorphisms in
dimension $\leq k+n+1$ such that $\H_q(X^{(n+1)})=0$ for $k<q\leq k+n+1$. Then
$g:X\to X'=\lim_n\{X\to X^{(n)}\}$ is a free map with $\H_q(X')=0$ for $q>k$ which
is an isomorphism in dimensions $\leq k$.

Form a co-cartesian diagram
\[
  \begin{tikzcd}
    X\ar[r,"f"]\ar[d,"g"'] & Y\ar[d,"g'"]\\
    X'\ar[r,"f'"] & Y'
  \end{tikzcd}
  \tag{7.4}
\]
and note that $g'$ is an isomorphism in dimension $\leq k+1$, since $g$ is.

Constructing a map $Y'\to Y''$ which is an isomorphism in dimension $\leq k+1$
and has $\H_q(Y'')=0$ for $q>k$, and replacing $Y'$ by $Y''$ we obtain a diagram
(7.4) where the vertical maps are free and isomorphisms in dimensions $\leq k+1$ and
$\H_q(X')=\H_q(Y')=0$ for $q>k$.

If $f$ is a $k$-equivalence, then as $g$, $g'$ are so is $f'$. Thus $f'$ is a weak
equivalence, hence a homotopy equivalence since $X'$ and $Y'$ are free. Thus
$F(g)$, $F(f')$, $F(g')$ are $k$-equivalences so $F(f)$ is.
\end{proof}

\begin{env}{Corollary}{7.5}
If $X$ is $k$-connected to is $\L F(X)$.
\end{env}

(7.6). For simplicial objects in an abelian category $\A$ the $0$-th homology functor
$\H_0:\sA\to\A$ is left adjoint to the functor $c:\A\to\sA$. This also holds for more
general categories, such as categories of universal algebras havig an underlying abelian
group law, and in particular for the category of ring-modules. Hence given a simplicial
ring and module $(R,X)$, the canonical adjunction map $(R,X)\to(c\H_0 R,c\H_0 X)$ gives
rise to a map $F(R,X)\to F(\H_0 R,\H_0 X)$ and hence to a canonical map
\[
  \H_0(F(R,X))\lra F(\H_0 R,\H_0 X).\tag{7.7}
\]
We shall say that $F$ is right exact if this map is always an isomorphism.
$F=S$, $\wedge$, and $\Gamma$ are all right exact because they are left adjoint
functors. For example if $F=S$ we have
\begin{align*}
  \Hom_{\Alg{\H_0 R}}(S^{\H_0 R}(\H_0 X),A)&=\Hom_{\Mod{\H_0 R}}(\H_0 X,A)\\
  &=\Hom_{\M_R}(X,cA)\\
  &=\Hom_{\Alg{\text{simp},R}}(S^R X,cA)\\
  &=\Hom_{\Alg{\H_0 R}}(\H_0(S^R X),A).
\end{align*}

(7.8). If $(B,M)$ is a ring-module let $L_q F(B,M)=\H_q(\L F(B,M))$. If $F$ is
right exact clearly
\[
  L_0 F(B,M)\simeq F(B,M).
\]

\begin{prop}{7.9}
There is a spectral sequence
\[
  E_{pq}^2=\H_p((L_q F)(R,X))\Lra\H_{p+q}(\L F(R,X))
\]
which when $F$ is right exact has the edge homomorphism
\[
  \H_n(\L F(R,X))\lra E_{n0}^2=\H_n(F(R,X))\tag{7.10}
\]
which is the map on homology induced by the canonical map $\L F(X)\to F(X)$.
\end{prop}
\begin{proof}
This spectral sequence is similar to the K{\"u}nneth spectral sequence th.6(b) of
[HA, II] and is constructed in pretty much the same way. We construct an exact sequence
in $\M_R$
\[
  \cdots\lra P_{(2)}\lra P_{(1)}\lra P_{(0)}\lra X\lra 0\tag{7.11}
\]
by recursion, letting $X_{(0)}=X$, $P_{(q)}\to X_{(q)}$ be a free resolution of
$X_{(q)}$, and $X_{(q+1)}=\ker(P_{(q)}\to X_{(q)})$. Let
$Q_{(\ast)}=N_{(\ast)}^{-1}(P_{(\ast)})$ be the simplicial object in $\M_R$ obtained by
applying the inverse of the normalization functor to the complex $P_{\ast()}$
([?, \textsection 3]). Then
\[
  Q_{(k)}=\textstyle\bigoplus_\eta P_{(t\eta)}
\]
where $\eta$ runs over all surjective morphisms with source $[k]$ and target $[t\eta]$.
From this we see that (i) $Q_{(k)}$ is a free ? module (ii) the inclusion
$P_{(0)}\to Q_{(k)}$, coming from $\eta=$ unique map: $[k]\to[0]$, is a homotopy
equivalence. Indeed by contruction $\H_\ast(P_{(k)})=0$ for $k>0$ hence $P_{(k)}$ is
contractible.

Now consider the bisimplicial abelian group $K_{pq}=F(R_q,Q_{(p)q})$ and the two
associated spectral sequences having the homology of the diagonal simplicial abelian
group $K_{nn}$ (for common abutment see [?] Satz.~2.15 or [?]). Using the property (ii)
we have
\[
  \H_q^v(K_{p\ast})=\H_q F(R,P_{(0)})
\]
for all $p$ hence
\[
  \H_p^h\H_q^v(K_{\ast\ast})=\begin{cases} 0,& p>0,\\
                                           \H_q F(R,P_{(0)}),& p=0.
                             \end{cases}
\]
Thus the spectral sequence with this as $E^2$ degenerates showing that the map
$F(R,P_{(0)})\to\Delta K$ is a weak equivalence. For fixed ?, the exactness of
(7.11) together with property (i) imply that $Q_{(\ast)n}$ is a free simplicial
$R_m$-module resolution of $X_m$, hence
\[
  \H_q^h(K_{\ast m})=(L_q F)(R_m,X_m)
\]
and
\[
  E_{pq}^2=\H_p^v\H_q^h(K_{\ast\ast})=\H_p((L_q F)(R,X))\Lra\H_{p+q}(\Delta K).
\]
\end{proof}

\begin{thebibliography}{2}
\bibitem{HA}
[HA] D.~Quillen, \emph{Homotopical algebra}.
\bibitem{GT}
[GT] M.~Artin, \emph{Grothendieck topologies}.
\end{thebibliography}

\end{document}

